%%%%%%%%%%%%%%%%%%%%%%%%%%%%%%%%%%%%%%%%%%%%%%%%%%%%%%%%%%%%%%%%%%%%%%%%%%%%%%%%
%	Latex Notes Template
%	Zach Neveu
%	zachary.neveu@gmail.com
%%%%%%%%%%%%%%%%%%%%%%%%%%%%%%%%%%%%%%%%%%%%%%%%%%%%%%%%%%%%%%%%%%%%%%%%%%%%%%%%

% Geometry, font
\documentclass[12pt, letter]{article}
\usepackage[margin=0.8in]{geometry}
\usepackage[T1]{fontenc}
\usepackage{fourier}
\usepackage{titling}
\setlength{\droptitle}{-5em} 
\usepackage[parfill]{parskip}
\usepackage{graphicx}
\graphicspath{{imgs/}}
\usepackage{hyperref}

% Math stuff
\usepackage{amssymb}
\usepackage{amsmath}
\usepackage{bm}

%acronyms
\usepackage{acronym}
\newacro{PL}{programmable logic}
\newacro{PS}{processing system}
\newacro{CLB}{configurable logic block}
\newacro{IOB}{input/output block}
\newacro{PSM}{programmable switch  matrix}
\newacro{PIP}{programmable interconnect point}
\newacro{LUT}{lookup table}
\newacro{MUX}{multiplexor}

% Code Highlighting
\usepackage{minted}
\usemintedstyle{solarizedlight}

\author{Zach Neveu}
\title{ Day 1: Intro }

\begin{document}
\maketitle

\subsection*{Organization}
\begin{itemize}
	\item Based on PYNQ framework: Python code in Jupyter, FPGA code in Python subset
	\item Blackboard has all kinds of fun extras, discussion boards etc.
	\item Using Vivado 2018.3 - version important, will error out if different
	\item 3 quizzes and a late midterm (quasi final)
	\item Second half of semester largely project based
	\item No final exam, slot will be used for project presentations
\end{itemize}

\subsection*{Project}
\begin{itemize}
	\item Application w/ hw and sw on Pynq.
	\item Compare HW speedup
	\item Try different interface types/memory allocations
	\item Topics can vary, ML, CV, cryptography, signal processing etc.
\end{itemize}

\subsection*{Topics}
\begin{itemize}
	\item What is an FPGA?
	\item Programming
	\item FPGA overlays
	\item Architecture
	\item Computer Arithmetic
	\item All of this subject to change a bit over the course
\end{itemize}

\subsection*{What's on an FPGA?}
\begin{itemize}
	\item Fabric - general purpose logic gates
	\item Micro blaze - soft-core processor
	\item Zynq: true hard-core processor, AXI interface to a bunch of \ac{PL}
	\item Zynq \ac{PS}: ARM A9 - 10-12 years old, but still widely used
	\item Using the Pynq Z2 for this class
\end{itemize}

\subsection*{Pynq}
\begin{itemize}
	\item Pynq written in Jupyter, uses IPython kernel
	\item Write hardware overlays and send to fabric. Call from Python.
	\item Overlays based on Pynq IP
	\item Overlays provide a good interface for software people to use custom hardware IP
\end{itemize}

\subsection*{What's on an FPGA p2}
\begin{itemize}
	\item \acp{CLB}
	\item \ac{IOB}
	\item \ac{PSM}
	\item \ac{PIP} - memory can control the interconnect - what connects to what?
	\item Everything based on memory
	\item Logic based on \acp{LUT}
	\item Many \acp{MUX} to decide what goes down wire
	\item \ac{LUT}: $2^{k}$ RAM cells for $k$ inputs, with a $2^{k} \to 1$ \ac{MUX} to choose which output to select
	\item \acp{MUX} implemented as tree of $2\to 1$ \acp{MUX}
	\item Memory values (truth table) chosen at compile time, \ac{MUX} path chosen by inputs
	\item 6 input \acp{LUT} are what's on the ZYNQ
	\item Each \ac{CLB} has 2 \acp{LUT} and 2 flip flops, muxed so you can use any combination of these
\end{itemize}
\end{document}
